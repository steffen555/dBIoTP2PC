\documentclass{article}
\usepackage[utf8]{inputenc}
\usepackage{graphicx}
\usepackage{textcomp}
\usepackage{amsfonts}
\usepackage{amsmath}
\usepackage{listings}
\usepackage{color}
\usepackage{bm}
\usepackage{soul,xcolor}
\setstcolor{red}

\newcommand*\xor{\oplus}

\title{Project Proposal}
\author{Thomas Hybel, 201303525 \\
Meinhard Dam, 201303531 \\
Steffen Strunge Mathiesen, 201407114}
\date{October 2017}

\begin{document}

\maketitle

\section*{Introduction}
For our project, we propose a distributed version of the card game Uno.

\section*{Use case}
Alice, Bob and Eve are having a party, but they are bored. They would like to
play a game of Uno, but Alice and Bob don't trust Eve -- she always cheats
when they play Uno.

Luckily, they all have access to computing devices connected to the same
network. So they launch the application Distributed Uno and join a game. Their
devices compute a distributed deck of cards which is shuffled. They are each
assigned a hand from the deck, letting them play the classic Uno game using
their devices.

The assignment of cards happens in a secure fashion, preventing Eve from
cheating. Since the application is distributed among many peers, there is no
central authority for Eve to attack either, so Alice and Bob feel at ease and
have a good time.


\section*{Description of how and why IoT/P2P will be used}
Our project mainly focuses on P2P to prevent cheating and avoid having a
central trusted authority. Specifically, P2P will be relevant for the
following features:
\begin{itemize}
\item Secure shuffling of the deck: no player should be able to choose the
order of cards in the deck.
\item Secure assignment of cards to players.
\item Commitment of all players to every move, ensuring agreement between
players and that moves cannot be undone at a later time.
\item Distributed high-score lists could be maintained.
\item A lobby system which easily lets players arrange games.
\end{itemize}

There will likely be many specific ways to cheat, and we plan to consider a
subset of these and prevent them using P2P techniques.

It is also possible to extend the project to have IoT aspects by, e.g., adding
special cards which interact with an IoT thing, such as a smartphone.

\section*{A sketch of the envisioned architecture}
We have not yet decided on the specifics of our architecture, but it will
incorporate a number of devices communicating with each other. Our
architecture will feature the following concepts:
\begin{itemize}
\item A normal card, with a number and a color
\item A special card, requiring a special action
\item A hand of cards
\item A deck of cards to pick from
\item A stack to deposit cards onto
\item Actions, such as picking up a card or passing on the turn
\item Rounds
\item A game which consists of rounds
\end{itemize}
There are various problems which we have not yet decided how to solve. Among
these are the following considerations:
\begin{itemize}
\item How should communication between players work? Multicast? A ring? Etc.
\item How should a shuffled deck be generated securely?
\item How can we ensure that players can't pretend to have other cards than
they do?
\item Should the deck be replicated on every peer, or on some of them, or be
stored distributed?
\end{itemize}
These open problems will be solved during our project work.


\section*{Weekly milestones}
\begin{itemize}
\item After week 43, we expect to have read relevant papers and figured out
how to solve many of the security problems in our architecture.
\item After week 44, we expect to have implemented secure deck shuffling and
hand assignment.
\item After week 45, we expect to have a working, mostly secure subset of Uno
with no special cards.
\item After week 46, we expect to have added at least some special cards to
our system.
\item After week 47, we expect to have added some additional feature(s) to our
system, such as a card based on an IoT-device, a lobby system, or such.
\end{itemize}



\end{document}

